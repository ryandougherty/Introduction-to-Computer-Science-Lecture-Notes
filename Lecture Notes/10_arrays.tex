\newsection{Arrays}

\subsection{Motivation}
Arrays are a very powerful tool in a programmer's arsenal. They allow managing many variables at once without much hassle.

\subsection{Arrays}
The typical structure for creating an array is:
\begin{center}
$<type>[] <name> = new <type>[<integer>];$
\end{center}
where $<integer>$ is greater than 0. For example, if we want to create an array of \verb|int| variables with 10 elements, we can construct it like this:
\begin{lstlisting}
int[] array = new int[10]; // 10 elements
\end{lstlisting}
Now how do we access or modify elements of an array? Here are some code examples on how to do so:
\begin{lstlisting}
int[] array = new int[100]; // 100 elements
// By default, each element is 0 (primitives) or null (objects)
// indexes are same as in String, etc. 0 - 99 indexes in this array
array[0] = 10; // set first element to 10
array[1] = 9; // set second element to 9
array[99] = 255; // set last element to 255
// array[100] = 100; // ArrayIndexOutOfBoundsException, because index is larger than the highest addressable one
// array[-1] = -1; // Also an ArrayIndexOutOfBoundsException

// Cannot assign/get different types or non-integer indexes:
// String s = array[10]; // compiler error
// array[10] = new String(); // compiler error
// System.out.println(array[2.5]); // compiler error

// However there is a ".length" attribute we can use (no parentheses):
System.out.println(array.length); // 100
// Therefore, we can set to the last element without a hardcoded number:
array[array.length-1] = 100;
// Getting an element is the same:
int n = array[56]; // get index 56 element
// We can loop over elements:
for(int i = 0; i < array.length; i++) {
     System.out.print(array[i] + " ");
}
// Or equivalently, the range-based for loop:
for(int i : array) { // "for each i in array"
     System.out.print(i + " ");
}
\end{lstlisting}

\subsection{Written Exercises}

\setcounter{counter}{1}
\begin{enumerate}[label={\arabic{counter}\addtocounter{counter}{1}}.]

\item What are the indices for the first and last positions of any array?

\item Immediately after instantiating a new array of primitives (ints, doubles, etc.), what fills the array? What about an array of objects?

\item What happens when you try to access an array element past the end of the array?

\item Use the following array x to answer the following questions:
\begin{table}[h]
\begin{tabular}{lllllll}
4 & 8 & 5 & 1 & 6 & 3 & 2
\end{tabular}
\end{table}
\begin{enumerate}
\item[a)]What value is given by x[1]?
\item[b)]What value is given by x[6]?
\item[c)]What value is given by x[7]?
\item[d)]What value is given by x.length?
\end{enumerate}

\end{enumerate}

\subsection{Programming Exercises}

\setcounter{counter}{1}
\begin{enumerate}[label={\arabic{counter}\addtocounter{counter}{1}}.]

\item Instantiate three arrays called x, y, and z of type int, String, and BankAccount (respectively), all of size 10.

\item Write a for-loop to sum all of the elements of an array x of type int.

\item Write a for-loop to double each element in an array x of type int.

\item Write code to store the largest number in an int array x into a variable called max.

\item Write code to count how many numbers in the array are strictly larger than 4, and store that total in a variable called total.

\item Write code to print out every other element in an array separated by tabs.

\item Write code to shift each number one place to the right (Note: there will be 2 copies of the 1st element when the code finishes).

\item Write code to print the contents of an array in reverse order, one element for each line.

\item Write a method called append that appends the two arrays passed as arguments and returns an array of type int as the result. For example, if the first array argument was \{1, 2, 3\}, and the second was \{4, 5, 6, 7\}, append returns \{1, 2, 3, 4, 5, 6, 7\}.

\item Write a method called findMin that returns the smallest element in an array that is passed as an argument. For example, if the array was \{4, 7, 9, 12, 8, 1, 5\}, the method would return 1.

\end{enumerate}