\newsection{Static}

\subsection{Motivation}
Now that we have created instance methods in the last chapter, we want to create class methods, which do not involve any instance state.

\subsection{Static}
The keyword \verb|static| in Java is meant to mean ``only one of these exists". If I have a static variable in a class (in the same place an instance variable will be), there will only be one copy of that variable, regardless of how many instances there are of that class.
\\ \\
For example, let's say we have a class called ``B" that has a \verb|static| field called ``numInstances" of type \verb|int| which is incremented every time the constructor is called. We would design the code like this:
\begin{lstlisting}
public class B {
     private static int numInstances = 0; // start with a value
     public B() {
          numInstances++;
     }
     public int getNumInstances() {
          return numInstances;
     }
}
\end{lstlisting}
If we call this from main:
\begin{lstlisting}
B b1 = new B();
System.out.println(b1.getNumInstances()); // 1
B b2 = new B();
System.out.println(b2.getNumInstances()); // 2
\end{lstlisting}
Now if we change the code to not have the instance variable be static, then we have:
\begin{lstlisting}
public class B {
     private int numInstances = 0; // remove "static"
     public B() {
          numInstances++;
     }
     public int getNumInstances() {
          return numInstances;
     }
}
\end{lstlisting}
And the outputs are:
\begin{lstlisting}
B b1 = new B();
System.out.println(b1.getNumInstances()); // 1
B b2 = new B();
System.out.println(b2.getNumInstances()); // 1
\end{lstlisting}
which is not the behavior that we want.
\\ \\
Now we can get a larger understanding of various conventions we have been using:
\begin{itemize}
\item The main method's use of \verb|static| in its signature shows that only 1 copy of that method can exist at a time in a Java program. Therefore, there is no ambiguity as to where execution starts.
\item For other code, such as \verb|Math.PI|, we can have an understanding as to possibly how it and some of its static methods are implemented:
\begin{lstlisting}
public class Math {
     // ...
     public final static double PI = 3.1415926;
     // ...
     public static double pow(double base,
          double exponent) { ... }
     // ...
     public static double sqrt(double num) { ... }
}
\end{lstlisting}

\end{itemize}

\subsection{Written Exercises}

\setcounter{counter}{1}
\begin{enumerate}[label={\arabic{counter}\addtocounter{counter}{1}}.]

\item What is a static variable? What is a static method?

\item Using the code below, how many copies of the variable number exist after instantiating 374 different AmazingClass objects?
\begin{lstlisting}
public class AmazingClass {
     private static int number;
     public AmazingClass(int a) {
          number = a;
     }
     public int twice() {
          number *= 2;
          return number;
     }
}
\end{lstlisting}

\item Using the code from above, what is the value of number after each of the following statements? (For each part, assume the preceding parts have already been executed).
\begin{lstlisting}
AmazingClass ac1 = new AmazingClass(3);
AmazingClass ac2 = new AmazingClass(7);
ac1.twice();
ac2.twice();
\end{lstlisting}

\end{enumerate}