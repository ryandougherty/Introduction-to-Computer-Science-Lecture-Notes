\newsection{Loops}

\subsection{Motivation}
With the last chapter, we now know how to execute conditional code. This chapter will greatly increase our powers in what we can do with programming, by being able to execute code many times without having to re-write (or copy and paste) code over and over.

\subsection{Introduction to Loops}
Loops are a way of executing some code an arbitrary number of times. What determines the number of times a loop executes is based largely on a condition, just like \verb|if| and \verb|switch| statements were. There are 3 kinds of loops in Java:
\begin{itemize}
\item \verb|while| loops
\item \verb|do|-\verb|while| loops
\item \verb|for| loops
\end{itemize}

\subsection{While loops}
\verb|while| loops are the simplest to understand kind of loop. The structure of one is:
\begin{center}
$while(condition) \{...\}$
\end{center}
The semantics are: the loop will continue as long as the condition evaluates to \verb|true|. The main reason \verb|while| loops are used is that we may not know the number of times looping occurs. If we do, we usually use a \verb|for| loop (see below). However, both \verb|while| and \verb|for| loops can be converted back and forth.
\\ \\
For example, let's say we are given a task of displaying a positive integer that the user enters, and continue doing so until the user enters 0. Setting up is easy:
\begin{lstlisting}
Scanner s = new Scanner(System.in);
while(/* something */) {
	// somehow take user input and display it
}
\end{lstlisting}
Let's initialize an \verb|int| variable, initialized to the first input a user gives:
\begin{lstlisting}
Scanner s = new Scanner(System.in);
System.out.println("Enter an integer:");
int input = s.nextInt();
while(/* something */) {
	// somehow take user input and display it
}
\end{lstlisting}
Now, we need to create the condition for the \verb|while| loop. We stop after the user enters 0, so let's make that condition (and display it):
\begin{lstlisting}
Scanner s = new Scanner(System.in);
System.out.println("Enter an integer:");
int input = s.nextInt();
while(input != 0) {
	System.out.println(input);
	// anything else go here?
}
\end{lstlisting}
This code seems good, but unfortunately we can encounter what is called an ``infinite loop", which does not stop looping. If we enter a non-zero integer, then we print that output. Then since there is nothing else to do in the loop body, we check the condition again, which evaluates to \verb|true|, prints out, loops back, prints out, etc. We need to get the user input again at the end of the loop:
\begin{lstlisting}
Scanner s = new Scanner(System.in);
System.out.println("Enter an integer:");
int input = s.nextInt();
while(input != 0) {
	System.out.println(input);
	System.out.println("Enter another integer:");
	input = s.nextInt(); // avoid infinite loop
}
\end{lstlisting}
Now our solution code for the example is correct.

\subsection{Do-While loops}
\verb|do-while| loops are the same as \verb|while| loops, but guarantee that the loop body will be executed at least once. Remember our earlier \verb|while| loop example: if we initially entered 0, then the loop's condition evaluates to \verb|false|, and does not execute the body. Therefore, it loops 0 times. 
\\ \\
The structure of a do-while loop is:
\begin{center}
$do \{ ... \} while(condition);$
\end{center}
Note: the semicolon after the last parentheses of the \verb|while|'s condition is necessary. For example, if we modify our example from above to be a \verb|do-while| loop, we guarantee that the user must enter at least one integer:
\begin{lstlisting}
Scanner s = new Scanner(System.in);
System.out.println("Enter an integer:");
int input = s.nextInt();
do {
	System.out.println(input);
	System.out.println("Enter another integer:");
	input = s.nextInt(); // avoid infinite loop
} while(input != 0);
\end{lstlisting}

\subsection{For loops}
\verb|for| loops are used when we know precisely how many times the loop will execute. However, they are equivalent in functionality to \verb|while| loops (i.e. one can construct a \verb|for| loop from a \verb|while| loop, and vice versa). All \verb|for| loops have the same structure:
\begin{center}
$for(initialization; condition; modification) \{...\}$
\end{center}
\begin{itemize}
\item Initialization - creating a variable, usually called the ``loop control" variable.
\item Condition - checking a condition, usually related to the loop control variable, against some value. The loop will continue to iterate (and do the modification step) as long as the condition is true.
\item Modification - modify, usually the loop control variable, at each iteration of the loop.
\end{itemize}
For example, let's say we are given the problem of summing all of the integers from 1 to n, where n is an integer $\ge 1$. An example solving this problem is the following:
\begin{lstlisting}
Scanner s = new Scanner(System.in);
int n = s.nextInt();
int sum = 0;
// put some kind of for loop here
System.out.println(sum);
\end{lstlisting}
Now we need to reason how to construct our \verb|for| loop. Let's initialize a loop control variable, called \verb|i|, to be 1. Then, for each iteration of the loop, add \verb|i| to \verb|sum|, and at each iteration, add 1 to \verb|i|. An ``unraveling" of the loop's logic will be this:
\begin{itemize}
\item sum = 0.
\item 1st iteration of loop: \verb|i| = 1, add to \verb|sum|, \verb|sum| = 1, add 1 to \verb|i| (\verb|i| = 2).
\item 2nd iteration of loop: \verb|i| = 2, add to \verb|sum|, \verb|sum| = 3, add 1 to \verb|i| (\verb|i| = 3).
\item 3rd iteration of loop: \verb|i| = 3, add to \verb|sum|, \verb|sum| = 6, add 1 to \verb|i| (\verb|i| = 4).
\end{itemize}
and so on. By this logic, we want the loop to stop at \verb|n|, inclusive. So, we need to have a condition of \verb|i <= n|. And on each loop, we execute \verb|i++|. Therefore, our final code will be:
\begin{lstlisting}
Scanner s = new Scanner(System.in);
int n = s.nextInt();
int sum = 0;
for(int i=1; i<=n; i++) {
	sum += i; // add to sum
}
System.out.println(sum);
\end{lstlisting}

\subsection{Written Exercises}

\setcounter{counter}{1}
\begin{enumerate}[label={\arabic{counter}\addtocounter{counter}{1}}.]

\item What are the 3 kinds of loops in Java?

\item What is the output of the following loop? How many times does the loop execute?
\begin{lstlisting}
int n = 979;
for (int j = 0; j <= n; j++) {
     System.out.println("Hello");
}
\end{lstlisting}

\item What is the output of the following loop? How many times does the loop execute?
\begin{lstlisting}
int j = 1;
int n = 5;
while (j <= n) {
     System.out.println("Hello");
     n--;
}
\end{lstlisting}

\item What is the output of the following loop? How many times does the loop execute?
\begin{lstlisting}
int n = 5;
for (int j = 1; j <= n; j += 3) {
     System.out.print("Hello ");
     int k = j;
     while (k < n) {
          System.out.println("Good Morning");
          k++;
     }
     j--;
}
\end{lstlisting}

\item What is the output of the following code?
\begin{lstlisting}
String name = "Richard M. Nixon";
boolean startWord = true;
for (int i = 0; i < name.length(); i++) {
     if (startWord)
          System.out.println(name.charAt(i));
     if (name.charAt(i) == ' ')
          startWord = true;
     else
          startWord = false;
}
\end{lstlisting}

\item What is the output of the following loop? How many times does the loop execute?
\begin{lstlisting}
int j = 1;
while (j <= 11) {
     System.out.println("Hello");
     j = j + 3;
}
\end{lstlisting}

\item What is the output of the following code?
\begin{lstlisting}
int n = 1, i = 1;
while (i < 7) {
     n = n * i;
     i += 2;
}
System.out.print(n);
\end{lstlisting}

\end{enumerate}

\subsection{Programming Exercises}

\setcounter{counter}{1}
\begin{enumerate}[label={\arabic{counter}\addtocounter{counter}{1}}.]

\item Write a loop that reads in \verb|int| values until the user enters 0 and prints out how many values entered are greater than 10.

\item Write a loop that will print out every other letter in a \verb|String| variable called \verb|str|. For example, if the \verb|String| was ``Hello There", then ``HloTee" will be printed.

\end{enumerate}