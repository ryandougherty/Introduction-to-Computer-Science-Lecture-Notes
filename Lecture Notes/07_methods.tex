\newsection{Methods}

\subsection{Motivation}
We have seen various ``methods" in the last chapter, but we want to improve on the functionality of our classes that we create. Therefore, we use methods to act on the objects we create. 

\subsection{Structure}
The typical structure of a method is:
\begin{center}
$<visibilitymodifier> <return type> <method name>(<optional parameters>) \{...\}$
\end{center}
For example, if I want to create a method called ``test" which takes a String parameter and returns nothing (i.e. \verb|void|), we would probably write it like this:
\begin{lstlisting}
public void test(String param) {
     // ...
}
\end{lstlisting}
Let's say that this method is in a class called ``A". Therefore, to call the method on the class, we would write (in main):
\begin{lstlisting}
A a = new A(); // or some other constructor
a.test("Hello");
\end{lstlisting}
We say that we ``call" the method ``test" on the object ``a" of type ``A". We will get a compilation error if instead of \verb|public|, we make the visibility modifier be \verb|private|. Also, a method can only have one return type, or \verb|void|, which means ``not returning a value," but is nevertheless a return type (just not a value). However, if we put \verb|void| as our return type, we can have a \verb|return| statement to immediately exit from the method:
\begin{lstlisting}
public void someMethod() {
     for (int i=0; i < 100; i++) {
          if (i == 57) {
               return; // will immediately stop execution of method
               // return 5; // compilation error, since we have void
          }
     }
}
\end{lstlisting}

\subsection{Methods Returning Values}
Methods can ``return" types, with the \verb|return| keyword. For example, if we want to implement a method that takes in 2 \verb|int| parameters and returns their product, we can do this:
\begin{lstlisting}
public int multiply(int num1, int num2) {
	return num1 * num2; // return this value
}
\end{lstlisting}
We can use this method, say in a class called \verb|C|, to call the method and use the value that is returned:
\begin{lstlisting}
C c = new C();
int result = c.multiply(3, 4); // result is 12
\end{lstlisting}
Since we specified \verb|int| as the return type of \verb|multiply|, we must return a \verb|int| at some point in the method. We will have a compilation error if we don't return a type when we say there will be, or returning an incorrect type.
\\ \\
Also, we must have a \verb|return| statement at all points that are reachable within the program (or combine them). For example, here is a method that will cause a compilation error:
\begin{lstlisting}
public int multiply(int num1, int num2) {
     if (num1 > 0) {
          System.out.print("num1 > 0");
          // compiler error here because no return value
     } else {
          return num1 * num2; // return this value
     }
}
\end{lstlisting}
We can see that if \verb|num1 > 0|, then we print that statement out, do not execute the \verb|else| clause, and continue. Since we can have a way through the method without returning, it causes a compilation error. We can mitigate this by adding a \verb|return| statement either inside the \verb|if| statement or at the end of the method (the ``catch-all" case):
\begin{lstlisting}
public int multiply(int num1, int num2) {
     if (num1 > 0) {
          System.out.print("num1 > 0");
          return 0; // fixed compilation error
     } else {
          return num1 * num2; // return this value
     }
     // can also put here: return 0;
}
\end{lstlisting}

\subsection{Written Exercises}

\setcounter{counter}{1}
\begin{enumerate}[label={\arabic{counter}\addtocounter{counter}{1}}.]

\item Write the output generated by the following program:
\begin{lstlisting}
public class Two {
     private double real, imag;
     public Two(double initReal, double initImag) {
          real = initReal;
          imag = initImag;
     }
     public double getReal() {
          return real;
     }
     public double getImag() {
          return imag;
     }
     public Two mystery(Two rhs) {
          Two temp = new Two(getReal()+rhs.getReal(), 
               getImag()+rhs.getImag());
          return temp;
     }
}
public class Test {
     public static void main(String[] args) {
          Two a = new Two(1.2, 3.4);
          Two b = a.mystery(a);
          Two c = b.mystery(b);
          System.out.println("1. " + a.getReal());
          System.out.println("2. " + a.getImag());
          System.out.println("3. " + b.getReal());
          System.out.println("4. " + b.getImag());
          System.out.println("5. " + c.getImag());
     }
}
\end{lstlisting}

\item Using these 2 classes, write the output of the following program:
\begin{lstlisting}
public class CDPlayer {
     private int totalTime;
     public CDPlayer() {
          totalTime = 0;
     }
     public int totalPlayTime() {
          return totalTime;
     }
     public void play(CDTrack aTrack) {
          totalTime += aTrack.getPlayTime();
     }
}
public class CDTrack {
     private String myTitle;
     private int myPlayTime, myTimesPlayed;
     public CDTrack(String trackTitle, int playTime) {
          myTitle = trackTitle;
          myPlayTime = playTime;
          myTimesPlayed = 0;
     }
     public int getPlayTime() {
          return myPlayTime;
     }
     public void wasPlayed() {
          myTimesPlayed++;
     }
     public String toString() {
          String result = "";
          int minutes = myPlayTime / 60;
          int seconds = myPlayTime % 60;
          result += myTitle + " " + minutes + ":" + seconds;
          result += " #plays = " + myTimesPlayed;
          return result;
     }
}
public class RunCDPlayer {
     public static void main(String[] args) {
          CDTrack t1 = new CDTrack("Day Tripper", 150);
          CDTrack t2 = new CDTrack("We Can Work it Out", 200);
          CDTrack t3 = new CDTrack("Paperback Writer", 138);
          CDPlayer diskPlayer = new CDPlayer();
          t1.wasPlayed();
          diskPlayer.play(t1);
          t2.wasPlayed();
          diskPlayer.play(t2);
          t1.wasPlayed();
          diskPlayer.play(t1);
          System.out.println(t1.toString());
          System.out.println(t2.toString());
          System.out.println(t3.toString());
          System.out.println("Total play time: " + 
               (diskPlayer.totalPlayTime() / 60) + ":" + 
               (diskPlayer.totalPlayTime() % 60));
	}
}
\end{lstlisting}

\end{enumerate}

\subsection{Programming Exercises}

\setcounter{counter}{1}
\begin{enumerate}[label={\arabic{counter}\addtocounter{counter}{1}}.]

\item Write a \verb|boolean| method called \verb|allDifferent| that takes 3 \verb|int| numbers and returns \verb|true| if they are all different, and \verb|false| otherwise.

\item Write a \verb|boolean| method called \verb|isPrime| that takes in an \verb|int| number and returns \verb|true| if it is \verb|prime|, and \verb|false| otherwise.

\end{enumerate}