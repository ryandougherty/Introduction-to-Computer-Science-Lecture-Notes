\section{Introduction}
%
Authors wishing to code their contribution
with \LaTeX{}, as well as those who have already coded with \LaTeX{},
will be provided with a document class that will give the text the
desired layout. Authors are requested to
adhere strictly to these instructions; {\em the class
file must not be changed}.

The text output area is automatically set within an area of
12.2\,cm horizontally  and 19.3\,cm vertically.

If you are already familiar with \LaTeX{}, then the
LLNCS class should not give you any major difficulties.
It will change the layout to the required LLNCS style
(it will for instance define the layout of \verb|\section|).
We had to invent some extra commands,
which are not provided by \LaTeX{} (e.g.\
\verb|\institute|, see also Sect.\,\ref{contbegin})

For the main body of the paper (the text) you
should use the commands of the standard \LaTeX{} ``article'' class.
Even if you are familiar with those commands, we urge you to read
this entire documentation thoroughly. It contains many suggestions on
how to use our commands properly; thus your paper
will be formatted exactly to LLNCS standard.
For the input of the references at the end of your contribution,
please follow our instructions given in Sect.\,\ref{refer} References.

The majority of these hints are not specific for LLNCS; they may improve
your use of \LaTeX{} in general.
Furthermore, the documentation provides suggestions about the proper
editing and use
of the input files (capitalization, abbreviation etc.) (see
Sect.\,\ref{refedit} How to Edit Your Input File).