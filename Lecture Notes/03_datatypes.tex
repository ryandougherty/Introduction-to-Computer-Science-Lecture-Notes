\newsection{Data Types}

\subsection{Motivation}
Now that we know how to compile and run programs, and print to the screen, we will bump up the complexity of our programs a little. In this chapter, we will be exploring data types, and how to do basic calculations.

\subsection{Types}
There are 8 ``primitive" (fundamental) data types in Java (we will explain constraints that they have later):
\begin{itemize}
\item \texttt{byte} - an integer with a small range.
\item \texttt{short} - same as \texttt{byte}, but a larger range.
\item \verb|int| - same as \verb|short|, but even larger range.
\item \verb|long| - same as \verb|int|, but even still larger range.
\item \verb|float| - a ``real" value, that has some precision.
\item \verb|double| - same as \verb|float| but with more precision.
\item \verb|boolean| - a type that takes one of two possible values: \verb|true|, or \verb|false|.
\item \verb|char| - a Unicode character. Think of \verb|a| or \verb|b| as \verb|char|s - any single character is a \verb|char|.
\end{itemize}

\noindent Now, we give the range for each of the primitive data types (all inclusive on both ends). These are all of the values that they can hold:
\begin{itemize}
\item \verb|byte|: -128 to +127.
\item \verb|short|: -32768 to +32767.
\item \verb|int|: $-2^{31}$ to $+2^{31}-1$.
\item \verb|long|: $-2^{63}$ to $+2^{63}-1$.
\item \verb|float|: $1.4 \times 10^{-45}$ to $3.4 \times 10^{38}$.
\item \verb|double|: $4.9 \times 10^{-324}$ to $1.7 \times 10^{308}$.
\item \verb|boolean|: \verb|true| or \verb|false|.
\item \verb|char|: 0 to 65535.
\end{itemize}
Now that we know the 8 fundamental data types, we want to know how to use them. To do that, we have to $declare$ a $variable$ with a type.

\subsection{Variable Declaration}
Like classes in the previous chapter, declaring variables follow the same basic structure, which is:
\begin{center}
$<Type>$ $<VariableName> = $ $<Value>;$
\end{center}
For example, if we want a \verb|int| variable called \verb|val| to have a value of 2, we would write the line of code as:

\begin{lstlisting}
int val = 2; // assign val to a value of 2
\end{lstlisting}

\noindent Also, we can declare several variables of the same type in the same line using a comma between the names:
\begin{lstlisting}
int val = 2, otherVal = 3;
\end{lstlisting}
Notes: 
\begin{itemize}
\item In a list of declarations, there is no need to declare the type twice (in fact, there will be a compiler error if this is done)
\item The convention in Java names is to follow camel casing: the first ``word" is lowercase, and the subsequent ``words" in the same variable are capitalized (see example above).
\end{itemize}

\noindent Now, there are only certain names that our Java variables can have. The rules are:
\begin{itemize}
\item The name cannot be a ``keyword" (any of the words in this document that are colored blue, such as \verb|int|, \verb|public|, etc.).
\item The first character must be a letter (either upper or lowercase), the underscore symbol, or a dollar sign (``\$"). However, a variable name cannot start with a number. 
\item Any character other than the first may be letters, underscores, the dollar sign, or also numbers. 
\end{itemize}

\noindent Here are some variable declarations using all primitive types:
\begin{lstlisting}
byte b = 1;
short s = 2;
int i = 300;
long l = 1000L; // note the 'L'
float f = 0.56f; // note the 'f'
double d = 523.6; // no suffix
boolean b = true;
char c = 'a'; // chars are surrounded in single quotes
\end{lstlisting}

\noindent After we have initialized a variable, we can also modify it without having to declare it again. Here is an example:
\begin{lstlisting}
int i = 300;
i = 301; // modifies i's value to be 301
// int i = 301; // this causes a compiler error 
		//(declare once)
\end{lstlisting}

\subsection{Primitive Operators}
Now that we know how to declare and modify values, we need to learn how to do various operations on them, such as addition, subtraction, multiplication, and division. An example showing all of them is the following (with comments):
\begin{lstlisting}
/* Addition */
// int works the same as byte and short
int i1 = 0, i2 = 3;
int i3 = i1 + i2; // adds i1's value (0) and i2's (3)

// doubles and floats work the same way:
double d1 = 2.5, d2 = 3.0;
float d3 = d1 + d2;
float f1 = 2.5f, f2 = 3.0f;
float f3 = f1 + f2;

// Can add across types, but is promoted to "most precise"
// If we add double + int, the result is a double
int i4 = 4;
double d4 = 5.3;
double d5 = d4 + i4;
// But we can also "cast" (with parentheses) to the type we want if it is less "precise" - this truncates the fractional part:
int casted = (int)(d4 + i4); // error without (int)

// cannot do operations on booleans (compiler error)
// boolean b1 = true + false;

/* Subtraction */
// Works with primitive types the same way as addition
int i5 = 0, i6 = 3;
int i7 = i5 - i6; // i6 has value -3

/* Multiplication */
// Multiplication is the same as addition in the same type
// However, we promote to "most precise" value if multiple types
int i8 = 5;
double d6 = 5.3;
double d7 = d6*i8; // promote i8 to a double valued 5.0

/* Division */
// Division is easy to understand except for when the numerator and denominator are both int/short/byte.
int i9 = 10, i10 = 3;
// What is i9 / i10? We expect 3.33333..., but we get 3 instead. Java does "integer division", where the numerator and denominator are both ints, which is truncating all the fractional parts.
int i11 = i9 / i10; // i11 has value 3.
// However, if we "cast" one of the values to a double, then we do not have this problem
double d8 = (double)i9 / i10; // cast i9 to a double of value 10.0
// d8 now has value 3.333333...
\end{lstlisting}

Another operation is called the mod operator (or ``remainder"):
\begin{lstlisting}
int i1 = 10, i2 = 3;
int i3 = i1 % i2; // i3 has value 1, since 10 remainder 3 = 1.
\end{lstlisting}

We can also ``shorten" times when we are modifying the same value with another. For example:
\begin{lstlisting}
int i = 5;
i += 3; // equivalent to: i = i + 3;
i -= 3; // equivalent to: i = i - 3;
i *= 3; // equivalent to: i = i * 3;
i /= 3; // equivalent to: i = i / 3;
\end{lstlisting}

For \verb|int| variables, there are 2 operations that are done very often, so they were put into the language: increment and decrement:
\begin{lstlisting}
int i = 2;
i++; // equivalent to: i += 1;
i--; // equivalent to: i -= 1;
\end{lstlisting}

\subsection{Boolean}

So how do we compare values? There are various operators (parentheses added for clarity - they are not needed):
\begin{lstlisting}
int i1 = 5, i2 = 6;
boolean larger = (i1 > i2); // false, since 5 < 6
boolean largerOrEqual = (i1 >= i2); // also false
boolean smaller = (i1 < i2); // true
boolean smallerOrEqual = (i1 <= i2); // true
boolean notEqual = (i1 != i2); // true
boolean equal = (i1 == i2); // false
// Note that == means compare value, = means assign
\end{lstlisting}

There are also operators to work with booleans: ``and", ``or", and ``not".
\begin{lstlisting}
/* And = && */
boolean t = true, f = false;
boolean andTrueTrue = t && t; // true
boolean andTrueFalse = t && f; // false
boolean andFalseTrue = f && t; // false
boolean andFalseFalse = f && f; // false
/* Or = || */
boolean orTrueTrue = t || t; // true
boolean orTrueFalse = t || f; // true
boolean orFalseTrue = f || t; // true
boolean orFalseFalse = f || f; // false
/* Not = ! */
boolean notTrue = !t; // false
boolean notFalse = !f; // true

// We can use multiple operators at once:
boolean myst = !(true && (true || false)); // false
\end{lstlisting}

What if we do not want our variable values to change? There is a way, with the \verb|final| keyword:
\begin{lstlisting}
final int NUM = 100; // convention: all capital letters for final variables
// NUM++; // compilation error
\end{lstlisting}

\subsection{String}
Now that we fully understand how primitive data types work, we can move on to using classes provided by the Java API (Application Programmer Interface). One such class is provided in the Java package \verb|java.lang|, called \verb|String|. This package is so often that Java automatically imports it for you, so the programmer does not have to do extra work.

\par A \verb|String| is a set of characters in between double quotes. Any character can be inside a string, except for a double quote. Examples are:
\begin{lstlisting}
String str1 = "I love";
String str2 = "Java";
String str3 = str1 + str2; // is now "I loveJava"
// str3 is called the "concatenation" of str1 and str2
int i1 = 3;
str3 = str1 + i1; // has value "I love3"
\end{lstlisting}

\noindent However, there is another way to declare a \verb|String| variable, and that is done using the Java keyword \verb|new|:
\begin{lstlisting}
String str1 = new String("I love"); // same as above
\end{lstlisting}

\verb|String| is the only Java class that can declare (also called ``instantiate") a variable without the \verb|new| keyword. All other Java classes must be instantiated with this keyword.

\par Now we need to see what we can do with these \verb|String| variables. As we covered in the last chapter, every Java class has one or more methods. The syntax for calling a method is:

\begin{center}
$variable.methodname(optional parameters)$
\end{center}

Note: \textbf{we cannot call methods on primitive data types} (see later chapters on classes and methods). Examples of various methods in Java for the \verb|String| class are:
\begin{lstlisting}
String str1 = "I love Java";
/*String.toUpperCase() and toLowerCase*/
String upper = str1.toUpperCase(); // "I LOVE JAVA"
String lower = str1.toLowerCase(); // "i love java"

/*String.replace(char c, char d)*/
// Replaces all instances of c with d
String repl = str1.replace('a', 'p'); // "I love Jpvp"

// We can even chain method calls
String myst = str1.replace('a', 'p').toUpperCase(); // "I LOVE JPVP"

/*String.length() - int which is # of characters*/
int len = str1.length(); // 11, since there are 11 characters

/*String.charAt(int n) - gets the nth character*/
// Strings are 0-indexed, so the first character is at index 0
// Error at runtime if input >= length()
char firstChar = str1.charAt(0); // has value 'I'
char secondChar = str1.charAt(1); // has value ' '
char lastChar = str1.charAt(str.length()-1); // 'a'

/*String.substring(int n, int m)*/
// gives a String that consists of index n through m, not inclusive on m
String substr1 = str1.substring(0, 3); // "I lo"
String substr2 = str1.substring(2, 6); // "love"

/*String.equals(String other)*/
// gives true (boolean) if the first String is exactly the same as the other
boolean equal1 = substr1.equals(substr2); // false
boolean equal2 = "Equal".equals("Equal"); // true
// We can also check for equality ignoring the case
boolean equal3 = "Equal".equalsIgnoreCase("eqUal"); // true

/*String.indexOf(char c)*/
// Returns index (int) of the first instance of c; otherwise, -1
int found1 = str1.indexOf('o'); // 4
int found2 = str1.indexOf('f'); // -1

/*String.compareTo(String other)*/
// Returns int showing lexicographic ordering of strings according to ASCII table (negative if first < second, 0 if equal, positive if first > second)
int c1 = "A".compareTo("B"); // negative
int c2 = "A".compareTo("b"); // more negative, lowercase has higher ASCII value
int c3 = "A".compareTo("A"); // 0
int c4 = "B".compareTo("A"); // positive
\end{lstlisting}

\subsection{Scanner}
Now that we know how to create and modify modify \verb|String| variables, it is important to know how to get input from the user so that we can make our programs usable. For that, we use the \verb|Scanner| class in the \verb|java.util| package.

\par However, since this class is not in the package \verb|java.lang|, we must $import$ it from the Java library, by putting an import statement at the beginning of our file:
\begin{lstlisting}
import java.util.Scanner; // import the Scanner class from the java.util package
// Equivalent: import java.util.*; // import everything from this package
public class SomeClass {
	// ...
}
\end{lstlisting}

\noindent So how do we use this class? We must initialize it with the \verb|new| keyword:
\begin{lstlisting}
Scanner s = new Scanner(System.in); // System.in means from the keyboard
\end{lstlisting}
\noindent This code creates an variable (often called ``object" in this case, since it is a class) in which we can call various methods on it to get input from the user. There are various ways of getting input from the user:

\begin{lstlisting}
Scanner s = new Scanner(System.in);
int x = s.nextInt(); // the first integer entered by the user (after pushing the enter/return key) will be put into x
// If an integer is not entered, then a run-time error will occur.
double y = s.nextDouble(); // if a user enters a double
String str1 = s.next(); // this is the next "word" (does not include spaces)
String str2 = s.nextLine(); // this is the entire line of characters before pressing the enter/return key
\end{lstlisting}

%\subsection{Example: Putting it all together}
%Suppose we were given the task of having a menu at a restaurant, with a menu of: 
%\begin{itemize}
%\item Hamburgers = \$2.75 each
%\item Cheeseburgers = \$3.25 each
%\item Drinks = \$2.00 each
%\end{itemize}
%And we allowed as many choices for each as the user wanted. Our task is to take in the user's input of how much of each item he/she wants, and calculate the total. Suppose the user asked for 2 hamburgers, 1 cheeseburger, and 3 drinks. Therefore, the total will be: $2.75 \times 2 + 3.25 \times 1 + 2.00 \times 3 = 14.75$.
%\\ \\
%First, we need to set up our class. Let's call it RestaurantMenu:
%\begin{lstlisting}
%public class RestaurantMenu { // begin class
%     public static void main(String[] args) { // main
%     }
%}
%\end{lstlisting}
%
%Now, we know we need to take input from the user. Therefore, we need to create a Scanner variable (and do not forget to import the \verb|java.util| package!):
%
%\begin{lstlisting}
%import java.util.Scanner; // get user input
%public class RestaurantMenu { // begin class
%     public static void main(String[] args) { // main
%          Scanner input = new Scanner(System.in);
%     }
%}
%\end{lstlisting}
%
%\noindent We know that there are 3 items on the board, and that the user will enter three integers. Therefore, we need to get these integers from the user. It may be helpful to print to the user how much of each item to input: 
%
%\begin{lstlisting}
%import java.util.Scanner; // get user input
%public class RestaurantMenu { // begin class
%     public static void main(String[] args) { // main
%          // Menu
%          System.out.println("Menu");
%          System.out.println("Hamburger - $2.75");
%          System.out.println("Cheeseburger - $3.25");
%          System.out.println("Drink - $2.00");
%          // Get Inputs from user
%          Scanner input = new Scanner(System.in);
%          System.out.println("# hamburgers?");
%          int numHamburgers = input.nextInt();
%          System.out.println("# cheeseburgers?");
%          int numCheeseburgers = input.nextInt();
%          System.out.println("# drinks?");
%          int numDrinks = input.nextInt();
%     }
%}
%\end{lstlisting}
%
%\noindent Now we need to include the price of each item in some way. It would be preferable to use \verb|final| variables to store the prices, since we can guarantee that the prices will never change (also preferable to put them near the beginning of the main method):
%\begin{lstlisting}
%import java.util.Scanner; // get user input
%public class RestaurantMenu { // begin class
%     public static void main(String[] args) { // main
%          // Prices
%          final double HAM_PRICE = 2.75;
%          final double CHEESE_PRICE = 3.25;
%          final double DRINK_PRICE = 2.00;
%          // Menu
%          System.out.println("Menu");
%          System.out.println("Hamburger - $2.75");
%          System.out.println("Cheeseburger - $3.25");
%          System.out.println("Drink - $2.00");
%          // Get inputs from user
%          Scanner input = new Scanner(System.in);
%          System.out.println("# hamburgers?");
%          int numHamburgers = input.nextInt();
%          System.out.println("# cheeseburgers?");
%          int numCheeseburgers = input.nextInt();
%          System.out.println("# drinks?");
%          int numDrinks = input.nextInt();
%     }
%}
%\end{lstlisting}
%
%\noindent The last two parts are to compute the total price based on the user input, and to print out the result to the user:
%\begin{lstlisting}
%import java.util.Scanner; // get user input
%public class RestaurantMenu { // begin class
%     public static void main(String[] args) { // main
%          // Prices
%          final double HAM_PRICE = 2.75;
%          final double CHEESE_PRICE = 3.25;
%          final double DRINK_PRICE = 2.00;
%          // Menu
%          System.out.println("Menu");
%          System.out.println("Hamburger - $2.75");
%          System.out.println("Cheeseburger - $3.25");
%          System.out.println("Drink - $2.00");
%          // Get Inputs from user
%          Scanner input = new Scanner(System.in);
%          System.out.println("# hamburgers?");
%          int numHamburgers = input.nextInt();
%          System.out.println("# cheeseburgers?");
%          int numCheeseburgers = input.nextInt();
%          System.out.println("# drinks?");
%          int numDrinks = input.nextInt();
%          // Calculate total price
%          double totalPrice = numHamburgers*HAM_PRICE;
%          totalPrice += numCheeseburgers*CHEESE_PRICE;
%          totalPrice += numDrinks*DRINK_PRICE;
%          System.out.println("Total Price: " + totalPrice);
%     }
%}
%\end{lstlisting}

\subsection{Math}
There is another class, called \verb|Math| (in the \verb|java.lang| package) that provides some useful functionalities:
\begin{lstlisting}
/* Square root */
double num1 = 105.0;
double d1 = Math.sqrt(num1);

/* PI - constant */
double pi = Math.PI; 

/* Math.{min, max} - minimum or maximum of 2 ints */
int num1 = 5, num2 = 3;
int max = Math.max(num1, num2); // 5
// also can be: Math.max(num2, num1);
int min = Math.min(num1, num2); // 3

/* Pow - take first double to power of second */
double d2 = 3.5, d3 = 4.3;
double d4 = Math.pow(d2, d3); // = 3.5^4.3
\end{lstlisting}


\subsection{Written Exercises}

\begin{enumerate}
\item Give the output of the following program:
\begin{lstlisting}
public class Example {
     public static void main(String[] args) {
          int y = 2, z = 1;
          z = y * 2;
          System.out.print(y + z);
     }
}
\end{lstlisting}

\item What will be the output of the following program?
\begin{lstlisting}
public class Example {
     public static void main(String[] args) {
          String s = new String("Arizona state university");
          char ch1 = s.toLowerCase().toUpperCase().charAt(0);
          char ch2 = s.toUpperCase().charAt(8);
          char ch3 = s.toUpperCase().charAt(s.length() - 1);
          System.out.println("ch 1 is: " + ch1);
          System.out.println("ch 2 is: " + ch2);
          System.out.println("ch 3 is: " + ch3);
     }
}
\end{lstlisting}

\item What will be the output of the following program?
\begin{lstlisting}
public class Example {
     public static void main(String[] args) {
          int num1 = 4, num2 = 5;
          System.out.println("4"+"5");
          System.out.println(num1+num2);
          System.out.println("num1"+"num2");
          System.out.println(4+5);
     }
}
\end{lstlisting}

\item Which of the following correctly invokes the method \verb|length()| of the \verb|String| variable \verb|str| and stores the result in \verb|val| of type \verb|int|?
\begin{lstlisting}
int val = str.length();
int val = length.str();
int val = length().str;
int val = length(str);
\end{lstlisting}

\item Evaluate each of the following expressions.
\begin{lstlisting}
String s = "Programming is Fun";
String t = "Workshop is cool";
System.out.println(s.charAt(0)+t.substring(3, 4));
System.out.println(t.substring(7));
\end{lstlisting}

\item Evaluate each of the following expressions, assuming \verb|j| is an \verb|int| with value 11, \verb|k| is an \verb|int| with value 3, and \verb|s| is a \verb|String| with value ``Ford Rivers".
\begin{lstlisting}
j / k
j % k
s.substring(1, 5)
s.length()
s.charAt(3)
\end{lstlisting}

\item True or False? The type \verb|String| is a primitive data type.

\item True or False? The type \verb|char| is a primitive data type.

\item Write the output of the following program:
\begin{lstlisting}
public class Question {
     public static void main(String[] args) {
          String str = "hello";
          System.out.println("abcdef".substring(1, 3));
          System.out.println("pizza".length());
          System.out.println(str.replace('h', 'm'));
          System.out.println("hamburger".substring(0, 3));
          System.out.println(str.charAt(1));
          System.out.println(str.equals("hello"));
          System.out.println("pizza".toUpperCase());
          System.out.println(Math.pow(2, 4));
          double num4 = Math.sqrt(16.0);
          System.out.println(num4);
     }
}
\end{lstlisting}

\item Write the output of the following program:
\begin{lstlisting}
public class Question {
     public static void main(String[] args) {
          String s1 = "Clinton, Hillary";
          String s2 = new String("Obama, Barack");
          System.out.println(s1.charAt(2));
          System.out.println(s1.charAt(s1.length()-1));
          System.out.println(s2.toUpperCase());
          System.out.println(s2.substring(
               s2.indexOf(",")+2, s2.length());
     }
}
\end{lstlisting}

\item What value is contained in the \verb|int| variable \verb|length| after the following statements are executed?
\begin{lstlisting}
length = 2;
length *= 9;
length -= 6;
\end{lstlisting}

\item What is the result of \verb|2/4| when evaluated in Java? Why?

\end{enumerate}

\subsection{Programming Exercises}

\begin{enumerate}

\item Write a Java program that asks the user for the radius of a circle and finds the area of the circle.

\item Write a Java program that prompts the user to enter 2 integers. Print the smaller of the 2 integers.

\end{enumerate}