\newsection{Introduction to Classes}

\subsection{Motivation}
Now that we understand how loops work, we want to be able to create our own classes! Just like the ones that we have been using, such as Scanner, String, Math, and others. 

\subsection{Class Structure}
A class follows the same format as we have been for our main programs. However, there does not have to be a main method in a class. The only requirement for main is that the class we run a program from contains a main method.

\par A class contains what are called ``instance variables", a constructor, and 1 or more methods. But before we can talk about these, we need to cover visibility modifiers.

\subsection{Visibility Modifiers}
We have been using \verb|public| when we create our programs so far. Now we will cover what this means, and what other visibility modifiers are. They are, for any variable/method/class:
\begin{itemize}
\item \verb|public| - any class can see it.
\item \verb|private| - only the class that contains it can see it.
\item \verb|protected| - only the class that contains it and any children that inherit from it can see it. Do not worry about this visibility modifier - we will not use it in this chapter.
\end{itemize}

\subsection{Instance Variables}
Instance variables are variables that exist inside a class when they are created. We usually write instance variables to have a visibility of \verb|private|, to enforce what is called ``encapsulation", which means that we hide data (the instance variables) that we do not want ``outsiders" to change. 

\par For example, let's create a BankAccount class:
\begin{lstlisting}
public class BankAccount {
	
}
\end{lstlisting}
In order to make an instance of the class, we need to create a ``constructor" for the class, which executes code inside it when we use the \verb|new| operator on the class, in the exact same way that we have used it for other Java-provided classes. Notes about the constructor:
\begin{itemize}
\item No return type
\item Optional parameters
\item Any number of constructors allowed as long as they differ in signature (number of arguments or different types of arguments).
\item The default constructor--with no arguments--is provided by Java if none is provided by the programmer.
\end{itemize}
For example, let's modify our \verb|BankAccount| class that has the name of the person with the account, and a starting balance. When we initialize our \verb|BankAccount| object in the main method, we will use the \verb|new| operator, which passes in the values supplied to \verb|name| and \verb|startAmount| below:
\begin{lstlisting}
public class BankAccount {
     // visibility of public - need to see outside of class	
     public BankAccount(String name, double startAmount) {
	
     }
}
\end{lstlisting}
Now what do we do with these parameters? We need a way of ``saving" them inside the class for later use. That is where instance variables come in. Let's create our instance variables with public visibility for now:
\begin{lstlisting}
public class BankAccount {
     public String personName;
     public double balance;
     
     // visibility of public - need to see outside of class	
     public BankAccount(String name, double startAmount) {
          personName = name; // set name
          balance = startAmount; // set amount
          // optional: use this operator
          // this.personName = name;
          // this references the BankAccount class when inside the class, and what comes after the dot references the instance variable with that name
     }
}
\end{lstlisting}
And in our main (or other that instantiates a BankAccount object) method:
\begin{lstlisting}
public class BankAccountTest {
     public static void main(String[] args) {
          BankAccount b = new BankAccount("Jane", 0.0);
     }
}
\end{lstlisting}
Since we have declared our instance variables as \verb|public|, we can modify the corresponding values of our instance to anything we want by using the ``dot" operator, in the same way we use \verb|Math.PI| on the \verb|Math| class:
\begin{lstlisting}
public class BankAccountTest {
     public static void main(String[] args) {
          BankAccount b = new BankAccount("Jane", 0.0);
          b.personName = "Another Name";
          b.balance = 500000;
     }
}
\end{lstlisting}
We can clearly see why this is not preferred. Therefore, we need to mark the visibility of our instance variables as private (and the operations above will result in a compile-time error):
\begin{lstlisting}
public class BankAccount {
     private String personName;
     private double balance;
     
     public BankAccount(String name, 
          double startAmount) {
          personName = name; // set name
          balance = startAmount; // set amount
     }
}
\end{lstlisting}
This is a good start, but we want to be able to deposit or withdraw some amount into balance. Therefore, we need to use ``setters" (also called ``mutators") and ``getters" (also called ``accessors"). Setters are about changing some internal value to a passed-in value, and getters are about getting internal values back to the user:
\begin{lstlisting}
public class BankAccount {
     private String personName;
     private double balance;
     
     public BankAccount(String name, 
          double startAmount) {
          personName = name; // set name
          balance = startAmount; // set amount
     }
     
     /*
     Getters have the form:
     public <Type> get<NameOfVar>() {
          return <NameOfVar>;
     }
     */
     
     // get the current balance
     public double getBalance() {
          return balance;
     }
     
      /*
     Setters have the form:
     public void set<NameOfVar>(<Type> other) {
          <Instance Var> = other;
     }
     */
     
     // set the balance to the input
     public void setBalance(double other) {
          balance = other;
     }
}
\end{lstlisting}
But this introduces the same problem as earlier - we can make our balance be any value we want. Therefore, we should get rid of the setter, and introduce a \verb|deposit| method, which adds to the balance with the input. We can leave the getter because it does not modify the value of \verb|balance|.
\begin{lstlisting}
public class BankAccount {
     private String personName;
     private double balance;
     
     public BankAccount(String name, 
          double startAmount) {
          personName = name; // set name
          balance = startAmount; // set amount
     }
     
     // get the current balance
     public double getBalance() {
          return balance;
     }
     
     public void deposit(double amount) {
          balance += amount;
     }
}
\end{lstlisting}
And in main, we can use these methods accordingly:
\begin{lstlisting}
public class BankAccountTest {
     public static void main(String[] args) {
          BankAccount b = new BankAccount("Jane", 0.0);
          b.deposit(50.0);
          System.out.print(b.getBalance());
     }
}
\end{lstlisting}

\noindent Notes about classes:
\begin{itemize}
\item A method (terminology to be covered in the next chapter) called \verb|toString|, which returns a \verb|String|, is commonly implemented among programmer-created classes. The returned \verb|String| is a textual description of the state of the class and instance variables. For instance, we could use our \verb|BankAccount| class and create a \verb|toString| method that returns:
\begin{verbatim}
Name: Jane
Balance: 50.0
\end{verbatim}
with the following code:
\begin{lstlisting}
public String toString() {
     // return textual description
     String toReturn = "";
     toReturn += "Name: " + personName;
     toReturn += "\n";
     toReturn += "Balance: " + balance;
     return toReturn;
}
\end{lstlisting}
And when the object is printed out, even without calling \verb|.toString()|, the textual description will be printed:
\begin{lstlisting}
// in main
BankAccount b = ...;
System.out.print(b); // calls toString() automatically
// equivalent to:
System.out.print(b.toString());
\end{lstlisting}
\textbf{Note}: Not providing a \verb|toString()| method in a class, when printing an instance, will print a hash value (i.e. random letters and numbers) instead of what is expected, along with the name of the class.
\end{itemize}

\subsection{Written Exercises}

\begin{enumerate}

\item Which of the following enforces Encapsulation?
\begin{enumerate}
\item[a)]Make instance variables private
\item[b)]Make methods public
\item[c)]Make the class final
\item[d)]Both a and b
\item[e)]All of the above
\end{enumerate}

\item Use the following class to answer the questions below:
\label{question:store}
\begin{lstlisting}
public class Store {
	private int quantity;
	private double price;

	public Store(int q, double p) {
		quantity = q;
		price = p;
	}

	public int getQuantity() {
		return quantity;
	}

	public void setPrice(double p) {
		price = p;
	}

	public double calcTotal() {
		return price * quantity;
	}
}
\end{lstlisting}
\begin{enumerate}
\item[a)]What is the name of the class?
\item[b)]List all instance variables of the class.
\item[c)]List all methods of the class.
\item[d)]List all mutators in the class.
\item[e)]List all accessors in the class.
\item[f)]List which method is the constructor.
\end{enumerate}

\item True or False? If no constructor is provided, then Java automatically provides a default constructor.

\item True or False? A method must have at least 1 return statement.

\end{enumerate}

\subsection{Programming Exercises}

\begin{enumerate}

\item For the \verb|Store| class in the Written Exercises above, do the following:
\begin{enumerate}
\item[a)]Write a mutator for the quantity.
\item[b)]Write an accessor for the price.
\item[c)]Write a line of code that will create an instance called \verb|videoStore| that has quantity 100 and a price of 5.99.
\item[d)]Call the \verb|calcTotal| method with the \verb|videoStore| object (from part c) to print out the total.
\end{enumerate}

\item Correct the following class definition if you think it will not work:
\begin{lstlisting}
public class Student {
     private String name, major;
     public Student() {
          name = "???";
          major = "xxx";
     }
     public Student(String n, String m) {
          n = name;
          m = major;
     }
     public String getMajor() {
          return m;
     }
     public String getName() {
          return n;
     }
}
\end{lstlisting}

\item Implement a class called \verb|AsuStudent|. The class should keep track of the student's name, number of classes registered, hours spent per week for a class (consider a student devotes the same amount of time for each of his/her classes per week). Implement a \verb|toString| method to show the name and number of classes registered by a student, a \verb|getName| method to return the name of the student, a \verb|getTotalHours| method to return the total number of hours per week, and a \verb|setHours| method to set the number of hours the student devotes for each class.

\end{enumerate}