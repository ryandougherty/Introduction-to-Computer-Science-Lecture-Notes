\newsection{Methods}

\subsection{Motivation}
We have seen various ``methods" in the last chapter, but we want to improve on the functionality of our classes that we create. Therefore, we use methods to act on the objects we create. 

\subsection{Structure}
The typical structure of a method is:
\begin{center}
$<visibilitymodifier> <return type> <method name>(<optional parameters>) \{...\}$
\end{center}
For example, if I want to create a method called ``test" which takes a String parameter and returns nothing (i.e. \verb|void|), we would probably write it like this:
\begin{lstlisting}
public void test(String param) {
     // ...
}
\end{lstlisting}
Let's say that this method is in a class called ``A". Therefore, to call the method on the class, we would write (in main):
\begin{lstlisting}
A a = new A(); // or some other constructor
a.test("Hello");
\end{lstlisting}
We say that we ``call" the method ``test" on the object ``a" of type ``A".

\subsection{Written Exercises}

\setcounter{counter}{1}
\begin{enumerate}[label={\arabic{counter}\addtocounter{counter}{1}}.]

\item Write the output generated by the following program:
\begin{lstlisting}
public class Two {
     private double real, imag;
     public Two(double initReal, double initImag) {
          real = initReal;
          imag = initImag;
     }
     public double getReal() {
          return real;
     }
     public double getImag() {
          return imag;
     }
     public Two mystery(Two rhs) {
          Two temp = new Two(getReal()+rhs.getReal(), 
               getImag()+rhs.getImag());
          return temp;
     }
}
public class Test {
     public static void main(String[] args) {
          Two a = new Two(1.2, 3.4);
          Two b = a.mystery(a);
          Two c = b.mystery(b);
          System.out.println("1. " + a.getReal());
          System.out.println("2. " + a.getImag());
          System.out.println("3. " + b.getReal());
          System.out.println("4. " + b.getImag());
          System.out.println("5. " + c.getImag());
     }
}
\end{lstlisting}

\item Using these 2 classes, write the output of the following program:
\begin{lstlisting}
public class CDPlayer {
     private int totalTime;
     public CDPlayer() {
          totalTime = 0;
     }
     public int totalPlayTime() {
          return totalTime;
     }
     public void play(CDTrack aTrack) {
          totalTime += aTrack.getPlayTime();
     }
}
public class CDTrack {
     private String myTitle;
     private int myPlayTime, myTimesPlayed;
     public CDTrack(String trackTitle, int playTime) {
          myTitle = trackTitle;
          myPlayTime = playTime;
          myTimesPlayed = 0;
     }
     public int getPlayTime() {
          return myPlayTime;
     }
     public void wasPlayed() {
          myTimesPlayed++;
     }
     public String toString() {
          String result = "";
          int minutes = myPlayTime / 60;
          int seconds = myPlayTime % 60;
          result += myTitle + " " + minutes + ":" + seconds;
          result += " #plays = " + myTimesPlayed;
          return result;
     }
}
public class RunCDPlayer {
     public static void main(String[] args) {
          CDTrack t1 = new CDTrack("Day Tripper", 150);
          CDTrack t2 = new CDTrack("We Can Work it Out", 200);
          CDTrack t3 = new CDTrack("Paperback Writer", 138);
          CDPlayer diskPlayer = new CDPlayer();
          t1.wasPlayed();
          diskPlayer.play(t1);
          t2.wasPlayed();
          diskPlayer.play(t2);
          t1.wasPlayed();
          diskPlayer.play(t1);
          System.out.println(t1.toString());
          System.out.println(t2.toString());
          System.out.println(t3.toString());
          System.out.println("Total play time: " + 
               (diskPlayer.totalPlayTime() / 60) + ":" + 
               (diskPlayer.totalPlayTime() % 60));
	}
}
\end{lstlisting}

\end{enumerate}

\subsection{Programming Exercises}

\setcounter{counter}{1}
\begin{enumerate}[label={\arabic{counter}\addtocounter{counter}{1}}.]

\item Write a boolean method called allDifferent that takes 3 int numbers and returns true if the numbers are all different, and false otherwise.

\item Write a boolean method called isPrime that takes in an int number, and returns true if the number is prime, and false otherwise.

\end{enumerate}