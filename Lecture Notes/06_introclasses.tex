\newsection{Introduction to Classes}

\subsection{Motivation}
Now that we understand how loops work, we want to be able to create our own classes! Just like the ones that we have been using, such as Scanner, String, Math, and others. 

\subsection{Class Structure}
A class follows the same format as we have been for our main programs. However, there does not have to be a main method in a class. The only requirement for main is that the class we run a program from contains a main method.
\\ \\
A class contains what are called ``instance variables", a constructor, and 1 or more methods. But before we can talk about these, we need to cover visibility modifiers.

\subsection{Visibility Modifiers}
We have been using \verb|public| when we create our programs so far. Now we will cover what this means, and what other visibility modifiers are. They are, for any variable/method/class:
\begin{itemize}
\item \verb|public| - any class can see it.
\item \verb|private| - only the class that contains it can see it.
\item \verb|protected| - only the class that contains it and any children that inherit from it can see it. Do not worry about this visibility modifier - we will not use it in this chapter.
\end{itemize}

\subsection{Instance Variables}
Instance variables are variables that exist inside a class when they are created. 

\subsection{Written Exercises}

\setcounter{counter}{1}
\begin{enumerate}[label={\arabic{counter}\addtocounter{counter}{1}}.]

\item Which of the following enforces Encapsulation?
\begin{enumerate}
\item[a)]Make instance variables private
\item[b)]Make methods public
\item[c)]Make the class final
\item[d)]Both a and b
\item[e)]All of the above
\end{enumerate}

\item Use the following class to answer the questions below:
\label{question:store}
\begin{lstlisting}
public class Store {
	private int quantity;
	private double price;

	public Store(int q, double p) {
		quantity = q;
		price = p;
	}

	public int getQuantity() {
		return quantity;
	}

	public void setPrice(double p) {
		price = p;
	}

	public double calcTotal() {
		return price * quantity;
	}
}
\end{lstlisting}
\begin{enumerate}
\item[a)]What is the name of the class?
\item[b)]List all instance variables of the class.
\item[c)]List all methods of the class.
\item[d)]List all mutators in the class.
\item[e)]List all accessors in the class.
\item[f)]List which method is the constructor.
\end{enumerate}

\item True or False? If no constructor is provided, then Java automatically provides a default constructor.

\item True or False? A method must have at least 1 return statement.

\end{enumerate}

\subsection{Programming Exercises}

\setcounter{counter}{1}
\begin{enumerate}[label={\arabic{counter}\addtocounter{counter}{1}}.]

\item For the Store class in the Written Exercises above, do the following:
\begin{enumerate}
\item[a)]Write a mutator for the quantity.
\item[b)]Write an accessor for the price.
\item[c)]Write a line of code that will create an instance called videoStore that has quantity 100 and a price of 5.99.
\item[d)]Call the calcTotal method with the videoStore object (from part c) to print out the total.
\end{enumerate}

\item Correct the following class definition if you think it will not work:
\begin{lstlisting}
public class Student {
     private String name, major;
     public Student() {
          name = "???";
          major = "xxx";
     }
     public Student(String n, String m) {
          n = name;
          m = major;
     }
     public String getMajor() {
          return m;
     }
     public String getName() {
          return n;
     }
}
\end{lstlisting}

\item Implement a class called AsuStudent. The class should keep track of the student's name, number of classes registered, hours spent per week for a class (consider a student devotes the same amount of time for each of his/her classes per week). Implement a toString method to show the name and number of classes registered by a student, a getName method to return the name of the student, a getTotalHours method to return the total number of hours per week, and a setHours method to set the number of hours the student devotes for each class.

\end{enumerate}